\documentclass[12pt,pdftex,oneside]{uqam_these}
\include{mes_macros}
\include{graphicx}

\begin{document}

La classe Connexion :

\begin{enumerate}

\item Description :  La classe ConnexionHttp permet de faciliter l'établissement
d'une connexion HTTP côté serveur. La ConnexionHttp permet d'ouvrir une
connexion TCP sur un port en particulier et d'attendre les requête qui y seront dirigées.
Les requêtes HTTP reçue d'un client sont ensuite encapsulées dans un objet
Requete. De la même façon, pour renvoyer une réponse à un client, il suffit
d'encapsuler la réponse HTTP dans un objet Reponse et la ConnexionHttp se
chargera de l'acheminer.

\includegraphics[width=\textwidth]{schema_connexion.jpeg}

\item Utilisation : 

Deux constructeurs distincts permettent de créer un objet Connexion
associé à un port donné. Si une adresse distante est spécifiée, le
nouvel objet tentera automatiquement d'établir la connexion. Sinon, il
attendra de recevoir une requête d'un client.

Utilisation du contructeur à un paramètre : 
\begin{verbatim}

maConnexion := Connexion.create( 80 );

\end{verbatim}

Utilisation du contructeur à deux paramètres : 

\begin{verbatim}

maConnexion := Connexion.create( '192.168.2.100', 80 );

\end{verbatim}

Si la Connexion a été créée avec une adresse, on peut tout de suite
considérer que le lien est établi et commencer à envoyer des
messages. Si elle a été créée sans adresse, on peut tout de suite
commencer à lire des messages. L'opération de lecture bloquera tant
qu'un client ne se sera pas connecté et n'aura pas envoyé un message.

Les opérations de lecture et d'écriture s'effectuent exactement comme
la lecture au clavier et l'affichage à l'écran par readln et writeln,
c'est-à-dire par le passage par référence d'une variable «~réceptrice~».

\begin{verbatim}

var
  chaineLue : String;
  chaineEnvoyee : String;
begin
  maConnexion.lireChaine( chaineLue );

  chaineEnvoyee := 'Bonjour le monde';
  maConnexion.ecrireChaine( chaineEnvoyee );
end;

\end{verbatim}

Tant que la connexion n'est pas fermée par le client ou le serveur,
les opérations de lecture et d'écriture se font toutes de et vers le
même hôte distant.

On peut forcer la fermeture d'une connexion grâce à la méthode
fermerConnexion. Si l'objet a été crée comme connexion serveur, il est
alors possible d'attendre la connexion d'un nouveau client en faisant
une nouvelle opération de lecture bloquante. Si l'objet a été créé
comme connexion client, une nouvelle tentative de connexion sera
effectuée à la première opération d'écriture.


\item Attributs :
  \begin{enumerate}
    
  \item port : Integer
        Le port sur lequel la connexion serveur attend les connexions client
        ou port sur lequel tenter d'établir une connexion à un serveur
        distant.

  \item adresse : String 
        Adresse du serveur distant avec lequel établir la connexion.
  \end{enumerate}

\item Constructeurs
  \begin{enumerate}
  \item create( uneAdresse : String; unPort : Integer );
        Crée un objet Connexion client qui se connectera sur le serveur spécifié par l'adresse et le port
        Le port doit être une valeur valide (0 < unPort < 65536) sinon une
        exception est lancée avec le message «~Port invalide~».
        \begin{enumerate}
        \item Paramètres
          \begin{enumerate}
          \item uneAdresse
                L'adresse du serveur distant
          \item unPort
                Le port sur lequel se connecter sur le serveur
          \end{enumerate}
        \end{enumerate}

  \item create( unPort : Integer );
        Crée un objet Connexion serveur qui écoutera sur le port spécifié
        Le port doit être une valeur valide (0 < unPort < 65536) sinon une
        exception est lancée avec le message «~Port invalide~».
        \begin{enumerate}
        \item Paramètres
          \begin{enumerate}
          \item unPort
                Le port sur lequel se connecter sur le serveur
          \end{enumerate}
        \end{enumerate}
  \end{enumerate}

\item Destructeur : 
Destructeur par défaut. Ferme toutes les connexions existantes.

\item Méthodes : 

  \begin{enumerate}
  \item lireChaine (var uneChaine : String );
        Lit une chaîne de caractère de l'ordinateur distant
        \begin{enumerate}
        \item Paramètres
          \begin{enumerate}
          \item uneChaine
                La variable qui contiendra la chaîne reçue
          \end{enumerate}
        \item Fonctionnement
          \begin{itemize}
          \item Si la connexion client n'existe pas, établie une connexion
          serveur en écoutant sur le port \emph{port}. Bloque jusqu'à
          ce qu'un client soit connecté.

          \item Lit des données du connecteur jusqu'à ce qu'une ligne
            complète soit reçue (terminée par CRLF).

          \item Si plus d'une ligne est reçue, affecte la
            première ligne à \emph{uneChaine} et conserve le restant pour le prochain appel.
          \end{itemize}

        \end{enumerate}

  \item ecrireChaine ( uneChaine:String );
        Envoie une chaîne de caractère à l'ordinateur distant
        \begin{enumerate}
        \item Paramètres
          \begin{enumerate}
          \item uneChaine
                La chaîne à envoyer. Elle doit se terminer par un retour de chariot (caractères 13 et 10)
          \end{enumerate}
        \item Fonctionnement
          \begin{itemize}
            \item Si la connexion client n'existe pas vérifie
              l'\emph{adresse}
            \item Si l'\emph{adresse} est vide, termine. Sinon, crée
              la connexion client selon \emph{adresse} et \emph{port}
            \item Envoie la chaîne \emph{uneChaîne}
          \end{itemize}
        \end{enumerate}

  \item fermerConnexion;
        Ferme la connexion.
        \item Fonctionnement
          \begin{itemize}
            \item Si la connexion client n'existe pas ou n'est pas
              établie, termine.
            \item Déconnecte la connexion client.
            \item Détruit l'objet connexion client.
          \end{itemize}
  \end{enumerate}
\end{enumerate}
\end{document}

